输入:给定任意一个coupon bond
输出:该coupon bond的present value, duration以及convexity
	从市场获得与当前coupon bond相似的其他bond的y,这里的相似主要指的是利率、到期时间等相似
	通过牛顿迭代法找到需要的$y_i$,具体进行几轮牛顿迭代法,找到几个$y_i$,取决于从当前时刻到到期,还要发放利息几次
	利用上文给出的公式,将$y_i$的具体值带入,求出PV,duration和convexity。

本计算方法的核心在于,我们的利率计算模型。也就是求到期收益率y。对于每一个到期收益率y,有一个现值PV一一对应。对于给定的一个coupon bond,我们在计算其现值PV时,可以有两种处理方法。Y随着时间的变化而变化,可以认为每天都有一个y值(但是不一定每天的y值都不相同,也可能有些日期的y值是相同的。)

法一
将未来每次发息日发放的利息,折现到当前时刻。例如,假设未来还将发息3次,分别在0.2年、0.7年、1.2年以后,那么我们需要计算出0.2年后对应的到期收益率y,0.7年后的到期收益率y以及1.2年以后的到期收益率y。计算这些y的方法,是通过与给定coupon bound相似的一些coupon bound或者zero coupon bound的y值,来拟合出一条到期收益率y与时间t的关系曲线。通过这种方式,我们将y和t之间的散点图变为了连续函数,使得我们可以得到任何一天的到期收益率y。从而通过将未来的利息及本金折现到当前时刻,得到当前时刻的现值PV。

法二
在法一中,我们需要多个y。这里,我们使用一个y来求得当前时刻的现值PV。因为y和PV是一一对应的关系。所以我们通过相似债券的y和PV的关系图,利用牛顿迭代法,直接得到某个到期收益率y对应的现值PV。那么如果知道我输入的coupon bond的在当前时刻y是多少呢?我们可以对一群相似的coupon bond的到期收益率进行加权平均来估计要求的coupon bond的到期收益率y。
