\subsection{直接使用Cython编译}
为了提升系统性能,我们的初步尝试是使用Cython来将原有Python程序转化为C语言并进行编译。
\paragraph{性能分析}在百万数据集上,调用C++加速。
\begin{table}[H]
\begin{adjustwidth}{-3cm}{-3cm}
\begin{center}
\begin{tabular}{|p{.8\textwidth}| p{.2\textwidth}|} \hline
技术手段 & 运行时间  \\ \hline
原始python代码 & 约50s  \\ \hline
使用Cython编译各函数 & 约38s  \\ \hline
\end{tabular}
\end{center}
\end{adjustwidth}
\end{table}


\subsection{使用静态变量后再编译为C语言}
由于Python本身并不需要声明变量类型,在使用Cython将原有Python程序转化为C语言的过程中,需要加入大量用于类型检查的代码。对系统的主要函数使用静态变量,能够有效减少翻译得到的C语言的代码量,以进一步提高系统的整体性能。

\paragraph{性能分析}在百万数据集上,调用Python多进程。
\begin{table}[H]
\begin{adjustwidth}{-3cm}{-3cm}
\begin{center}
\begin{tabular}{|p{.8\textwidth}| p{.2\textwidth}|} \hline
技术手段 & 运行时间  \\ \hline
原始python代码 & 约50s  \\ \hline
使用Cython编译各函数 & 约38s  \\ \hline
使用Cython编译各函数并使用静态变量 & 约26s\\ \hline
\end{tabular}
\end{center}
\end{adjustwidth}
\end{table}

\subsection{使用Python多进程}
很显然,使用Python多进程在理论上可以加快代码的运行速度。

\paragraph{性能分析}在百万数据集上,调用Python多进程。
\begin{table}[H]
\begin{adjustwidth}{-3cm}{-3cm}
\begin{center}
\begin{tabular}{|p{.6\textwidth}| p{.2\textwidth}|p{.2\textwidth}|} \hline
技术手段 & 运行时间  &运行时间\\ \hline
原始python代码 & 约50s & 约13s \\ \hline
使用Cython编译各函数 & 约38s & 约11s \\ \hline
使用Cython编译各函数并使用静态变量 & 约26s& 约7.8s\\ \hline
\end{tabular}
\end{center}
\end{adjustwidth}
\end{table}

\subsection{使用分布式处理}
以上三种加速方法均是针对单机进行加速,在第四种方法中,我们尝试利用分布式地方法进行加速。
\begin{table}[H]
\begin{adjustwidth}{-3cm}{-3cm}
\begin{center}
\begin{tabular}{|p{.2\textwidth}| p{.8\textwidth}|} \hline
使用框架 & parallel python,它是一个基于python的并行计算以及分布式集群计算框架  \\ \hline
缺点& 没有负载均衡,节点之间工作分配不均匀,加速效果较差
  \\ \hline
\end{tabular}
\end{center}
\end{adjustwidth}
\end{table}