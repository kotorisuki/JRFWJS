%% LaTeX2e file `mytable1.tex'
%% generated by the `filecontents' environment
%% from source `Main' on 2017/12/25.
%%
  \begin{longtable}{lX}
%

    \hline
    名称 & 含义  \\ \hline
    Interest rate 利率 $r$& 利率,又叫利息率,是衡量利息高低的指标。是一定时期内利息额和本金的比率。利率=利息/本金。\\ \hline
    Discount factor 贴现因子 r &一般来说,当利率为r时,承诺T年之后支付R美元的现值是R美元/$(1+r)^T$。因此,即使没有通货膨胀,将来1美元的价值也小于现在1美元的价值,必须按某一数额贴现,该数额取决于利率的高低和收到货币的时间长短。其中$1/(1+r)^T$被称为未来T时期的货币的贴现因子. \\ \hline

    cash flow 现金流  &现金流量是现代理财学中的一个重要概念,是指企业在一定会计期间按现金收付实现制,通过一定经济活动(包括经营活动、投资活动、筹资活动和非经常性项目)而产生的现金流入、现金流出及其总量情况的总称。即:企业一定时期的现金和现金等价物的流入和流出的数量。\\ \hline
    Coupon 息票  & 息票一词来自英文coupon,原指旧时的债券票面的一部分,债券持有人可将其剪下,在债券付息日携至债券发行人处要求兑付当期利息。现在发行的债券多采用电子化形式,但票面利率(coupon rate)仍被用来表示债券的利率。\\ \hline
    Zero coupon bound 无息债券 & 无息债券 (Zero Coupon Bond) 债券无附设任何利息回报,发行机构以债券票面值在到期日偿还债券本金,故无息债券市价必定给予较票面值较大折让。 我国一次还本付息债券可视为无息债券.无息债券是指采用以复利计算的一次性付息方式付息的债券。无息债券又称“无息票债券”。按面值折扣发行,到期按面值十足还本的债券。\\ \hline

    Present Value 现值 $PV$ & 现值( Present value),指资金折算至基准年的数值,也称折现值、也称在用价值,是指对未来现金流量以恰当的折现率进行折现后的价值。指资产按照预计从其持续使用和最终处置中所产生的未来净现金流入量折现的金额,负债按照预计期限内需要偿还的未来净现金流出量折现的金额。\\ \hline
    Price 债券的市场价格  $P=v/{(1+y/f)^n} $& 债券在市场上交易时的价格。\\ \hline
    Face Value 票面价值  $v$& 债券面值是指债券发行时所设定的票面金额,它代表着发行人借入并承诺于未来债券到期日,偿付给债券持有人的金额。由于贴息债券的购买价低于债券面值,此外,债券发行有溢价发行和折价发行,因此,债券面值和投资债券的本金不一定相等。\\ \hline

    Yield 收益率 $y$ & 市场上同类债券的年化收益率,年化利率是通过产品的固有收益率折现到全年的利率。\\ \hline
    Payment Frequency 付息频率  $f$ & 一年支付利息次数。\\ \hline
    Time 付息次数  $n$& 累计计算利息次数。\\ \hline

    duration 久期 $D={dP/dy} \cdot {1/P}$ & 它是以未来时间发生的现金流,按照目前的收益率折现成现值,再用每笔现值乘以现在距离该笔现金流发生时间点的时间年限,然后进行求和,以这个总和除以债券目前的价格得到的数值就是久期。概括来说,就是债券各期现金流支付所需时间的加权平均值。\\ \hline
    convexity 凸性 $C={(d^2{P})/({dy}^2 )} \cdot {1/P}$& 久期描述了价格-收益率曲线的斜率,凸性描述了价格/收益率曲线的弯曲程度。凸性是债券价格对收益率的二阶导数。\\ \hline

  \caption{基本概念与定义}
  \label{tab:sys.param}
  \end{longtable}
